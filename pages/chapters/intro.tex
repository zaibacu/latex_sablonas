\chapter{ĮVADAS}

Pramogų industrija dažnai susiduria su paprastu, bet sunkiai atsakomu klausimu
- ar tai patiks žmonėms. Kadangi tai stipriai siejasi su pelnu, šiame abstraktume
reikia įžvelgti kažką apčiuopiamo, turinčio vienokią ar kitokią skaitinę išraišką.

Televizijos atveju, šią išraišką suteikia reitingai. Reitingai tinkami laidoms
ir serialams, nes galima prognozuoti žiūrovų kiekio kitimą, tačiau tai neveiksminga
kai kalbama apie kino filmus, nes jokios prognozės suformuoti negalima.

Metrika, į kurią galima atsigręžti, prognozuojant filmo žiūrovų skaičių - kino kritikai.
Kino kritikai, be išsamios apžvalgos, pateikia ir skaitinius įverčius. Palyginus
kritiko įverčius su jau rodytų filmų reitingais, galima daryti reitingų prognozes
naujiems filmams.

Vis tik kritikų įvertinimai turi dvi esmines problemas:
\begin{enumerate}
  \item Subjektyvumas. Nepaisant siekiamo profesionalumo, kritiko subjektyvumas
  visada turės įtakos recenzijai.
  \item Išlavintas skonis. Tai kas patinka išlavintam skoniui, nebūtinai patiks masėms.
  O patikti masėms yra pagrindinis tikslas.
\end{enumerate}

Dėl šių problemų, ilgainiui atsirado matas, kuris siekia parodyti masės nuomonę.
Vienas iš pionierių - CinemaScore, kompanija kurią 1979 metais įkūrė Ed Mitz.
Kompanijos atsiradimo istorija siejama su įkūrėjo patirtimi, kai jis, perskaitęs
kino filmo recenziją, nutarė jį pažiūrėti, tačiau liko nusivylęs. Kompanijos veikla
prasidėjo apklausiant žmones kurie neseniai matė kino filmą, siekiant suteikti filmui
konkretų įvertį \cite{cinema_score}.

Šiandien tai yra plačiai pripažįstamas matas, jį galima rasti tokiose populiariose svetainėse kaip
\url{http://imdb.com} ir \url{http://rottentomatoes.com}. Dėl didžiulio informacijos kiekio,
šie įverčiai yra ganėtinai tikslūs ir vertingi. Pagrindinė problema iškyla kai norime
prognozuoti konkrečiai auditorijai, pavyzdžiui, Lietuvai. Elementarūs kultūriniai
skirtumai iškreipia statistiką, o filtruoti duomenis siekiant to išvengt turime mažai galimybių.
\textit{IMDB} leidžia atskirti tik Jungtinių valstijų auditoriją nuo viso likusio pasaulio,
o \textit{Rotten Tomatoes} nepalaiko auditorijų filtravimo. Išeitis būtų naudotis
lokaliu reitingu, pavyzdžiui, \url{http://obuolys.lt}, tačiau tada susiduriame su kita problema
- nepakankamu informacijos kiekiu.

Faktas, kad didelę dalį interneto duomenų sudaro nestruktūrizuoti duomenys. Tikslų
kiekį įvardinti sunku, nes ir patys mokslininkai dėl to nesutaria \cite{unstructured_data}.
Pasinaudojus dalimi šių duomenų, orientuotų į kino filmus, galima gauti ganėtinai daug
žiūrovų įverčių. Šiuo metu, vien \url{http://linkomanija.net} turi virš 400 tūkstančių komentarų
apie kino filmus. Tinkamai panaudojus komentarų informaciją, galima gauti žiūrovų įvertį
tinkamą Lietuvos auditorijai.

Šio darbo tikslas yra aprašyti metodiką ir sukurti prototipą, leisiantį gauti žiūrovų įvertį
iš internetinių komentarų. Tikslui pasiekti, reikės atlikti žingsnius:

\begin{enumerate}
  \item Duomenų surinkimas ir anotavimas, pirminė analizė
  \item Eksperimentinis metodų tyrimas ir tobulinimas
  \item Prototipo projektavimas
  \item Prototipo realizavimas ir testavimas
\end{enumerate}

Metodika bus universali, todėl tiks įvairioms kalboms. Vis tik prototipas bus apmokytas
skaičiuoti įverčius remiantis lietuviška auditorija.
