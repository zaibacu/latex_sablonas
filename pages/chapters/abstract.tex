\chapter*{SANTRAUKA}
\addcontentsline{toc}{section}{SANTRAUKA}

\begin{tabular}{p{6cm} p{10cm}}
Autoriaus vardas ir pavardė:  & Šarūnas Navickas\\
Darbo pavadinimas:	          & Metodika ir prototipas automatiniam kino filmų žiūrovų įverčio skaičiavimui\\
Darbo vadovas:		            & Prof. Tomas Krilavičius\\
Pristatymas:				          & 2016-05-24\\
Puslapių skaičius:			      & 42\\
Lentelių skaičius:			      & 12\\
Paveikslų skaičius:			      & 20\\
Priedų skaičius: 			        & 5\\
Originalios PĮ apimtis:		    & 1400\\
\end{tabular}

\vspace{1.5cm}

Straipsnis aprašo kino filmų žiūrovų įverčio skaičiavimo temą.
Pateikiama metodika ir ją realizuojantis prototipas.
Metodika yra paremta sentimentų analize taikant mašininio mokymo algoritmus
ant internetinių komentarų apie kino filmus.
Įvairių teksto transformavimo metodų bei mašininio mokymo algoritmų rezultatai yra palyginti tarpusavyje.
Tyrimas atliktas taikant lietuvių kalbai, tačiau dauguma metodų tinkami visoms kalboms.

\newpage

\chapter*{ABSTRACT}
\addcontentsline{toc}{section}{ABSTRACT}

\begin{tabular}{p{6cm} p{10cm}}
Author of term paper: 		& Šarūnas Navickas\\
Full title of term paper:	& Technique and Prototype for Automatic Movie Audience Score Calculation\\
Supervisor:					      & Prof. Tomas Krilavičius\\
Presented at:				      & 2016-05-24\\
Number of pages:			    & 42\\
Number of tables:			    & 12\\
Number of pictures:			  & 20\\
Number of appendixes:		  & 5\\
Original program size:    & 1400\\
\end{tabular}

\vspace{1.5cm}

Article covers topic of audience score computation for movies.
It describes technique and defines prototype to implement it. 
Technique is based on applying machine learning algorithms for sentiment
analysis on internet comments about movies. 
Results from multiple text transformation methods and machine learning algorithms
are compared. The research is done on lithuanian language, however most of the methods
applies to any language. 
